\section{Experiments} % Sections can be created in order to organize your presentation into discrete blocks, all sections and subsections are automatically printed in the table of contents as an overview of the talk
%------------------------------------------------

% \begin{frame}
% \frametitle{Toy Data - Densely Sampled}
% \begin{figure}
%     \centering
%     \begin{subfigure}{.33\textwidth}
%       \centering
%       \includegraphics[width=.3\linewidth]{img/information_gain_developing.gif}
%       \caption{InformationGain}
%       \label{fig:sub1}
%     \end{subfigure}%
%     \begin{subfigure}{.33\textwidth}
%       \centering
%       \includegraphics[width=.3\linewidth]{img/ib10_developing.gif}
%       \caption{Information Bottleneck, $\beta=10$}
%       \label{fig:sub2}
%     \end{subfigure}
%     \begin{subfigure}{.33\textwidth}
%         \centering
%         \includegraphics[width=.3\linewidth]{img/ib5_developing.gif}
%         \caption{Information Bottleneck, $\beta=5$}
%         \label{fig:sub2}
%       \end{subfigure}
%     \caption{Effect of $\beta$ in densely sampled setting}
%     \label{fig:test}
%     \end{figure}
% \end{frame}

\begin{frame}
    \frametitle{Experiments}
    \begin{center}
        \textbf{\href{http://localhost:8888/notebooks/ib_trees.ipynb}{Open notebook on localhost}}\\
        \small{or} \\
        \textbf{\href{https://colab.research.google.com/github/nielsrolf/InformationBottleneckTree/blob/master/ib_trees.ipynb}{Open notebook on google colab}}
    \end{center}

\end{frame}