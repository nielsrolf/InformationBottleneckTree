\section{Decision Trees}
\begin{frame}
   \frametitle{Decision Trees}
   Decision tree inducers provide an algorithm to solve classification and regression problems.
   \begin{itemize}
      \item  A binary decision tree $T$ splits the data: $s_T(x) = x_{i_T} \leq t_{T}$ unless $T$ is a leaf
      \item If $T$ is a leaf, it predicts $T(x) = c_T$ for some constant $c_T$
      \item If $T$ is not a leaf, it creates two subtrees $T_{left}$ and $T_{right}$, and predicts:
      \[   
      T(x) = 
            \begin{cases}

               T_{left}(x) &\quad\text{if}\:  s_T(x) = \text{TRUE} \\
               T_{right}(x) &\quad\text{if}\:  s_T(x) = \text{FALSE} \\
            \end{cases}
      \]
      \item This defines the capacity of the model
   \end{itemize}
\end{frame}


\begin{frame}
   \frametitle{Decision Trees - Example}
   \begin{center}
      \includegraphics[height=150px]{img/true_vs_learned_regulartree.png}
   \end{center}
   \textit{Left:} Training data of an artificial classification problem. 
   \textit{Right:} Learned function of a decision tree
\end{frame}


\begin{frame}
   \frametitle{Decision Trees - Learning}  
   \begin{itemize}
   \item Fitting a tree is an optimization problem
   \item Many exact solutions are NP hard: e.g. finding a minimal tree that fits the data
   \item Instead of exact solutions: greedy algorithms (bottom up vs top down)
   \item Different loss functions have been proposed: \newline
   InformationGain, Gini Index,  Likelihood-Ratio Chi–Squared Statistics, DKM Criterion, Gain Ratio, ...
   \item This project evaluates the Information Bottleneck as loss function 
   \end{itemize}
\end{frame}


\begin{frame}
   \frametitle{Decision Trees - Top Down}  
   TODO Algorithm pseudocode
\end{frame}